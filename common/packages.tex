
\usepackage{cmap}
\usepackage[T2A]{fontenc}
\usepackage[utf8]{inputenc}         % file encoding
\usepackage{substitutefont}         % so we can use fonts other than those specified in babel
\usepackage[english]{babel}         % default language for document
\usepackage[scaled=0.925]{XCharter} % Подключение русифицированных шрифтов XCharter
\usepackage[bitstream-charter]{mathdesign} % Согласование математических шрифтов

\usepackage{
    %amsfonts,
    mathtools,
    mathtext,
    cite,
    enumerate,
    float,
    textcomp    
}                                   % some packages 
%\usepackage[font={small}]{caption}
\usepackage{hhline}                 % beautiful links
\PassOptionsToPackage{hyphens}{url}
\usepackage{hyperref}               % beautiful links
\usepackage{pdflscape}              % landscape pages
\usepackage{longtable}              % longtable support
\usepackage{multirow}               % vertical cell in table
\usepackage{bigstrut}               % big strut
\usepackage{array}                  % additional cell aligh
\usepackage{indentfirst}            % first line indent
\usepackage{gensymb}                % symbols
\usepackage{caption}

%\usepackage[nottoc]{tocbibind} %  do we need bibliography in toc
%----------------------------------------------
%\usepackage{mhchem}        
%===============  Font for tables  ============
%\let\oldtabular\tabular
%\renewcommand{\tabular}{\small\oldtabular}

\hypersetup{colorlinks=true,
    linkcolor=blue,
    citecolor=blue, 
    filecolor=blue, 
    urlcolor=blue, 
    pdftitle={Cinelerra GG Infinity Manual}, 
    pdfauthor={Cinelerra Authors},
    pdfsubject={Video Editing}, 
    pdfkeywords={Cinelerra, Good Guy}
} % pdf properties
\usepackage[pdftex]{graphicx}       % do we need some figures in our pdf 
\graphicspath{{images/}}            % path to images
\usepackage{tikz}                   % drawing package
\usepackage{nameref}                % use \nameref{} to set reference to chapter neme.

%----------------------------------------------------------------------
\usepackage{listings}               % include code 
\lstset{                            % begin settings
  %language=R,                      % the language of the code
  inputencoding=utf8,
  basicstyle=\ttfamily\footnotesize,         % the size of the fonts that are used for the code
  numbers=left,                     % where to put the line-numbers
  numberstyle=\tiny\color{black},   % the style that is used for the line-numbers
  stepnumber=1,                     % the step between two line-numbers. If it's 1, each line
                                    % will be numbered
  numbersep=5pt,                    % how far the line-numbers are from the code
  %backgroundcolor=\color{white},   % choose the background color. You must add \usepackage{color}
  showspaces=false,                 % show spaces adding particular underscores
  showstringspaces=false,           % underline spaces within strings
  showtabs=false,                   % show tabs within strings adding particular underscores
  frame=lines,                      % adds a frame around the code
  %frame=single,                    % adds a frame around the code
  rulecolor=\color{black},          % if not set, the frame-color may be changed on line-breaks within not-black text (e.g. commens (green here))
  tabsize=2,                        % sets default tabsize to 2 spaces
  captionpos=b,                     % sets the caption-position to bottom
  breaklines=true,                  % sets automatic line breaking
  breakatwhitespace=false,          % sets if automatic breaks should only happen at whitespace
  title=\lstname,                   % show the filename of files included with \lstinputlisting;
                                    % also try caption instead of title
  keywordstyle=\color{blue},        % keyword style
  commentstyle=\color{gray},        % comment style
  stringstyle=\color{black},        % string literal style
  %backgroundcolor=\color{green!10},
  escapeinside={\%*}{*)},           % if you want to add a comment within your code
  extendedchars=true,
  %keepspaces = true                %!!!! spaces in comments
  texcl=true,
  postbreak=\mbox{\textcolor{red}{$\hookrightarrow$}\space},
  morekeywords={*,...}              % if you want to add more keywords to the set
}
%---------------------------------------------------------------------------
\makeatletter
\renewcommand{\@biblabel}[1]{#1.}
%---------------------------------------------------------------
%
\usetikzlibrary{                    % Libraries for TiKz
    positioning,
    arrows,
    shapes,
    shadows
} 
\usepackage{wrapfig}                % Wrapping figures

\usepackage{enumitem}               % custom lists

\usepackage[colorinlistoftodos,textsize=tiny]{todonotes}    % todo package
\setlength{\marginparwidth}{1.6cm}                          % fix left margin for todo

\usepackage[intoc]{nomencl}                         % glossary package
\makenomenclature                                           % make glossary

%\usepackage{noto}

