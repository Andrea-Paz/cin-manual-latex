\chapter{Introduction}%
\label{cha:introduction}

Cinelerra is a software program NLE, Non-Linear Editor, that provides a way to edit, record, and play audio or video media. 
It can also be used to retouch photos.

This manual covers Cinelerra-GG Infinity version. 
The author of the original Cinelerra, Adam Williams, as well as many different people worked on Cinelerra over the years. 
The software and this manual were merged in from various sources and each person is to be thanked and commended for their efforts. 
Numerous software modifications were made by William Morrow. 
These are all under GPLv2+ license. 
Refer to: \url{https://www.gnu.org/licenses/old-licenses/gpl-2.0-standalone.html}


\textbf{This is a copy of the header from the original source code.}
\begin{lstlisting}[numbers=none]
/*
* CINELERRA
* Copyright (C) 1997-2012 Adam Williams <broadcast at earthling dot net>
*
* This program is free software; you can redistribute it and/or modify
* it under the terms of the GNU General Public License as published by
* the Free Software Foundation; either version 2 of the License, or
* (at your option) any later version.
*
* This program is distributed in the hope that it will be useful,
* but WITHOUT ANY WARRANTY; without even the implied warranty of
* MERCHANTABILITY or FITNESS FOR A PARTICULAR PURPOSE. See the
* GNU General Public License for more details.
*
* You should have received a copy of the GNU General Public License
* along with this program; if not, write to the Free Software
* Foundation, Inc., 59 Temple Place, Suite 330, Boston
\end{lstlisting}

\textbf{This is a copy of the information in the Cinelerra-CV manual.}

Copyright c 2003, 2004, 2005, 2006 Adam Williams - Heroine Virtual Ltd.
Copyright c 2003, 2004, 2005, 2006, 2007 Cinelerra CV Team.

This manual is free; you can redistribute it and/or modify it under the terms of the GNU General
Public License as published by the Free Software Foundation; either version 2 of the License, or
(at your option) any later version.

This document is distributed in the hope that it will be useful, but WITHOUT ANY WAR-
RANTY; without even the implied warranty of MERCHANTABILITY or FITNESS FOR A
PARTICULAR PURPOSE. See the GNU General Public License for more details.

You should have received a copy of the GNU General Public License along with this program;
if not, write to the Free Software Foundation, Inc., 51 Franklin St, Fifth Floor, Boston, MA 02110, U

\section{Cinelerra Overview}%
\label{sec:cinelerra_overview}

Presented briefly here is an overview of Cinelerra-GG Infinity and information provided in this manual. 
The GG version of Cinelerra has been improved for \emph{stability}, \emph{modernized} to accommodate the
\emph{current state} of Linux software, enhanced with additional \emph{basic features}, and enriched with \emph{new features} imagined by dedicated users and then implemented by professional programmers.

\begin{description}
    \item[Website] \url{https://www.cinelerra-gg.org}\\
        The website for the cinelerra-gg software is a good place to start for information, help, and documentation. 
        It is professionally maintained and continuously updated with more language
        translations ongoing.
    \item[Stability] ~\\
        Software programs that “just work” are a \#1 priority in order to be of use for producing quality videos.
        A large amount of time has been invested in debugging problems and resolving crashes. 
        And in a continuous process to do so, a chapter on Troubleshooting is included in order to easily provide sufficient information for users to capture issues and crashes so that they can be analyzed and quickly fixed to avoid repeat problems.
    \item[Modernization] ~\\
        Artistic creativity has been applied to modernize the Cinelerra-GG plugin icons for video and audio to even include the ffmpeg plugins. 
        The Cinfinity set of plugin icons come in square or roundish versions --- your choice. 
        In keeping up with the current expectation of users for a certain “look and feel”, a very modern Neophyte theme recent addition provides an alternative to the already existing 9 themes. 
        These 10 themes give the user the choice to get the look they like best for their own eyes.
    \item[Cerrent and up-to-date] ~\\
        For today’s software, included thirdparty libraries are kept up to date in a timely manner and effort is made to always used a relatively recent version of FFmpeg, if not the latest. 
        This is a big deal because there is a whole set of separate programmers continuously working diligently to cover all of the old and newly created formats. 
        Thus Cinelerra programmers can be dedicated to working solely on Cinelerra rather than just trying to keep up with many formats.
    \item[FFmpeg usage integration]~\\
        By using FFmpegwith Cinelerra there is the advantage that users can directly convert videos via pre- and post-processing, without the need for command lines to be executed manually before or afterwards.
    \item[Import and Export formats] ~\\
        Listed here are only a few of the supported Import and Export formats:
        \begin{itemize}
            \item  several standard native formats, such as mpeg, ac3, flac, exr and jpeg/png/ppm/tiff sequences
            \item FFmpeg’s over 400 decoders and 150 encoders accessible from within Cinelerra to include:
                mp4, mkv, mpeg, mov, m2ts, mp3, dvd, ogg, theora, prores, tiff, webm, flac, opus, vorbis,
                quicktime (div, dnxhd, jpeg, mjpeg, mp4, rle, v308, v410), h264 & h265 usage, avc, hevc,
                and recently released AV1 and WebP
            \item raw image format for over 700 supported cameras, courtesy Dave Coffin DCraw
        \end{itemize}
\end{description}




