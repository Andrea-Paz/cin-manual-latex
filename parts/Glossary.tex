\nomenclature{\textbf{Alpha channel}}{an extra channel in color models for storing information about transparency. If a pixel has a value of 0\% as its alpha channel, it is completely transparent or invisible; a value of 100\% as its alpha channel provides a fully opaque pixel.}
    
\nomenclature{\textbf{Arm}}{when you \textit{arm} a track, it is enabled so fully affected by editing operations. Tracks are armed with an \texttt{Arm Track} button in the patchbay of the timeline.}
 
\nomenclature{\textbf{Arrow mode}}{formally known as Drag and Drop editing mode. Drag and drop editing is a quick and simple way of working in Cinelerra, using mainly the mouse.  In this mode, you move \textit{things} by dragging and dropping these resources or objects.}

\nomenclature{\textbf{Aspect ratio}}{in reference to an image, this is the geometry of the data as in the ratio of its width to its height. There is the display aspect ratio which is the physical display and may not always be square.  Then there is the pixel aspect ratio or geometry of the picture.}

\nomenclature{\textbf{Assets}}{the representation of media loaded in the Resources Window and on the timeline.  These assets are the objects of the editing. You can think of an asset as the original file that was loaded from your operating system’s disk but it also represents clips and nested clips.  These are data objects in the program that contains all the parameters it takes to read or write that media.}

\nomenclature{\textbf{Audio offset}}{audio displacement applied to the playback position to synchronize playback with other tracks.}

\nomenclature{\textbf{Automatic keyframes}}{every time you tweak a key-framable parameter it “automatically” creates a keyframe on the timeline. Since automatic keyframes affect many parameters, it is best enabled just before you need a keyframe and disabled immediately after.  Effects are only keyframable in automatic mode because of the number of parameters in each individual effect.  Enable automatic keyframe mode by enabling the automatic keyframe toggle on the timeline editpanel.}

\nomenclature{\textbf{Autos}}{short term for automation keyframes.  Autos are created by clicking on an "automation curve" represented by a colored line, to establish the time position for the new keyframe anchor point. They are usually drawn as a colored square or a symbol on a media track.}

\nomenclature{\textbf{Bezier curve}}{a parametric curve used in computer graphics.  In image manipulation programs, bezier curves are used to model smooth curves.}

\nomenclature{\textbf{Bitrate}}{represents the amount of information that is stored per unit of time of a recording. The higher the bitrate, the higher the quality.}

\nomenclature{\textbf{Brightness}}{human perception of the amount of light emitted by a source; overall lightness or darkness of an image.  In video signals it is represented by luma. The measure of brightness is value.}

\nomenclature{\textbf{Buffer}}{a region of memory used to temporarily hold output or input data, used when there is a difference between the rate at which data is received and the rate at which it can be processed, or in the case that these rates are variable. Buffers are allocated by various processes to use as input queues, etc. A simplistic explanation of buffers is that they allow processes to temporarily store input in memory until the process can deal with it.}

\nomenclature{\textbf{Bug}}{a defect in the program that causes it to not function correctly thus creating some kind of error.  There are basically 2 kinds of bugs – doing something it should not or not doing something it should.}

\nomenclature{\textbf{Canvas}}{the place on the compositor where the final video is displayed. It can be imagined as the canvas of a painter or as the screen of a movie theater.}

\nomenclature{\textbf{Chroma}}{the signal used in many video systems to carry the color information of the picture separately from the accompanying luma signal. Luma represents the achromatic image without any color (like a black and white picture), while the chroma components represent the color information.}

\nomenclature{\textbf{Clip}}{a short segment of media which was usually part of a longer recording.  A section of a project consisting of a set of timeline selected edits treated as an independent object.}

\nomenclature{\textbf{Clipping}}{to set a data value above an threshold to be equal to that threshold.}

\nomenclature{\textbf{Codec}}{contraction of \textbf{co}der and \textbf{dec}oder; used to generally compress media and decompress to reproduce it. Results are a trade-off between bitrate and quality.}

\nomenclature{\textbf{Color correction}}{Same as color grading. Some people use it with slightly different meanings: minimizing error between the image and the original object. Corrects exposure and color dominants. There is the primary CC, which covers the entire frame; and the secondary CC, which isolates and intervenes only on specific parts of the frame.}

\nomenclature{\textbf{Color grading}}{Same as color correction. Some people use it with slightly different meanings: Correct the color and contrast of a shot to personalize it and make it more nicer or communicative. An example is the Teal \& Orange, which is widely used in Hollywood movies.}

\nomenclature{\textbf{Color model}}{an abstract mathematical model describing the way colors can be represented as an ordered list of numbers, typically as three values or color components (for example, RGB).}

\nomenclature{\textbf{Concatenate}}{in File$\rightarrow$Load insertion strategy, means to load from disk or copy from the timeline edits belonging to different media onto the same set of tracks, one right after the other.}

\nomenclature{\textbf{Compression}}{in Cinelerra’s dialogs means compression format. See Codec.}

\nomenclature{\textbf{Context menu}}{a pop-up menu.}

\nomenclature{\textbf{Control point}}{as applied to Bezier curves, either the \textit{end} points or \textit{way points}.  Controls the derivative at that end or way point of that curve.  Vector of the curve point derivative.}

\nomenclature{\textbf{CR2}}{the RAW image format of Canon’s digital cameras. In Cinelerra, it is used loosely to refer to any brand of camera raw images or still pictures, as handled by Dave Coffin’s included code.}

\nomenclature{\textbf{Crop}}{rectangularly trim edges to remove unwanted material.}

\nomenclature{\textbf{Cut}}{delete or trim to remove material or insert a break so that the media on the timeline is separated into 2 pieces.  Cut, when used as a noun can mean the same as an edit.}

\nomenclature{\textbf{Cut and Paste editing}}{an editing mode on the timeline.  You select the cut and paste editing mode by enabling the I-beam toggle on the control bar at the top of the main program window. The I-beam cursor pointer on the timeline is why this mode has the nickname of I-beam.  In this mode you can copy edits in the same track, copy from different tracks in the same instance, start a second instance of Cinelerra and copy from one instance to the other or load a media file into the Viewer and copy from there.  There are many other operations that can be done.}

\nomenclature{\textbf{Deinterlace}}{the process of converting interlaced video (a sequence of fields) into a non-interlaced form (a sequence of frames).  Deinterlacing downsamples/blends the refreshes that reconstruct images with the least motion damage.  See the definition of Interlace for an explanation of interlacing.}

\nomenclature{\textbf{Disarm}}{when you “disarm” a track, it is disabled so not affected by editing operations. Disarming a track protects it from most changes and operations. Tracks are disarmed with an "Arm Track" button in the patchbay of the timeline.}

\nomenclature{\textbf{Drag and Drop editing}}{drag and drop editing is a quick and simple way of working in Cinelerra using mainly the mouse.  In this mode, you move “things” by dragging and dropping these resources or objects. To enable, click on the arrow icon in the timeline editpanel.  This arrow is why this mode has the nickname of arrow mode.}

\nomenclature{\textbf{Edit(s)}}{is a fragment of media on a single track of the timeline.}

\nomenclature{\textbf{Edit Decision List (EDL)}}{an XML text file that contains all the project settings and locations of every edit. Instead of media, it contains pointers to the original media files on disk. If you open your .xml project file with a text editor, you will see some of the terminology used in Cinelerra in reference to the editing you perform.}

\nomenclature{\textbf{Edit panel}}{the second line on the timeline which shows icons that represent transport controls and editing functions.  It is sometimes called the Control Bar.}

\nomenclature{\textbf{Field}}{in interlace, the scan of every second line. An interlaced video frame consists of two sub-fields taken in sequence, each sequentially scanned at odd and then even lines of the image sensor.}

\nomenclature{\textbf{File format}}{generally refers to a block network stream format that multiplexes multiple audio/video streams to create presentation output.}

\nomenclature{\textbf{Filter}}{a program to process a data stream, often it is used in referring to effects or plugins.  Many people use this in referencing the included FFmpeg plugins since that is what it is called in ffmpeg.}

\nomenclature{\textbf{Footage}}{in filmmaking and video production, footage is the raw, unedited material as it had been originally filmed by a movie camera or recorded by a video camera, which usually must be edited to create a motion picture, video clip, television show or similar completed work.}

\nomenclature{\textbf{Frame}}{any image in a sequence of images that form an animated video.}

\nomenclature{\textbf{Gadget}}{a composition of widgets.  See widget definition.}

\nomenclature{\textbf{Gamma}}{color and contrast shaping using an exponential math function to enhance or retard color contrast, either color or brightness.}

\nomenclature{\textbf{GDB}}{Gnu DeBugger, a Linux program for testing and some help in finding the errors in another program.}

\nomenclature{\textbf{Git}}{is a distributed version control system for tracking changes in source code during software development. It is fast and easy to work with git repositories for collaboration among several developers on the same software project.}

\nomenclature{\textbf{GPU}}{stands for the Graphics Processor Unit of a computer graphics board.  For Cinelerra, direct decoding or encoding using the GPU via hardware acceleration often reduces CPU usage.}

\nomenclature{\textbf{GUI}}{Graphical User Interface allows people to interact with the computer by manipulating graphical icons, visual indicators or widgets, along with text labels or text navigation to represent the information and actions available to a user. This is in contrast to Command Line Interface (CLI).}

\nomenclature{\textbf{GUICAST}}{Cinelerra’s GUI library, made from scratch by Heroine Virtual Ltd.}

\nomenclature{\textbf{Handle}}{is a graphical grab point to adjust any number of different graphical parameters, such as an edit position or a mask curvature.  The handle becomes active when approached within a certain range, then you will generally see a modified cursor indicator.  This is the hot point to grab and click to use it.}

\nomenclature{\textbf{HSV}}{Hue, Saturation, and Value is a color model that is often used in place of the RGB color model. In using this color model, a color is specified, then white or black is added to easily make adjustments.}

\nomenclature{\textbf{Hue}}{that aspect of a color described with names such as yellow, red, blue.  Hue also defines mixtures of two pure colors like "red-yellow" or "yellow-green".}

\nomenclature{\textbf{Image list}}{a text file with a specific format containing a list of absolute paths for the still images of a sequence plus additional information like file format, framerate and image resolution. Image lists are human readable and editable. Once loaded in the timeline, image lists behave like a video clip. In Cinelerra they can be used to load multiple images belonging to the same scene as a single video asset. Cinelerra can render video clips to image lists (a text file + multiple still images).}

\nomenclature{\textbf{I-beam mode}}{formally known as Cut and Paste editing mode where a lot of the operations performed on your video are with various copy commands.  See Cut and Paste editing definition.}

\nomenclature{\textbf{Index file}}{an .idx, .mkr or .toc file built by Cinelerra in the .bcast5 directory of your home folder in order to quickly, and with less cpu involved, seek into big media files, for skipping, faster playback and drawing waveforms and picons. Index files are not human readable.  If an index file for an asset is already built, it is not recreated – this saves a lot of time. However if you switch from a native format to using ffmpeg, or vice versa, you should always rebuild the index. The number of index files to keep can be set by the user and you can easily delete all of them at once from the Settings→Preferences, Interface menu.}

\nomenclature{\textbf{Insertion point}}{a flashing hairline mark that vertically spans the timeline. It marks the starting place of the next operation to be performed.}

\nomenclature{\textbf{Interlace}}{(extended definition included here because it is no longer a commonly used technique). Originally media was conditioned for presentation on a cathode ray tube (CRT), like an old-fashioned television set.  The image sample was scanned using an image orthicon tube and vacuum tube oscillators. The result is very imprecise and too slow for the human eye.  Interlacing, which is scanning the image by fields, first all the even and then all the odd lines, uses phosphor persistence to create a smooth presentation effect and reduce the required signal bandwidth.  Today we use a progressive scan at a higher framerate to avoid these issues.}

\nomenclature{\textbf{Interpolation}}{a method of inventing data using keyframe way points on a curve.  The shape of the curve represents the type of the interpolation, for example piecewise linear or bezier. In Cinelerra interpolation is the process that inserts many new values in between the user defined keyframes to give a smoother result.}

\nomenclature{\textbf{Jitter}}{broken sound or video, frequently due to quantimization encoding, resulting in data loss.}

\nomenclature{\textbf{Keyframe}}{a blob of parameter data associated to a position on the timeline.  It represents a certain value set by the user at a certain point in the timeline and is used as input to some edit functions such as a fade, an effect, or a compositing parameter.  Keyframes are described in detail in the Keyframes section, chapter 8.}

\nomenclature{\textbf{Keyframing}}{a very convenient technique for creating smooth dynamic changes by assigning values to parameters at specific moments in time and letting Cinelerra interpolate the values in between.}

\nomenclature{\textbf{Latency}}{refers to a short period of delay, usually measured in milliseconds, between when an audio signal enters a system and when it emerges.}

\nomenclature{\textbf{Locale}}{a set of parameters that defines your language, country and other location related preferences that you want to see in your GUI. Usually a locale identifier consists of at least a language identifier and a region identifier. Example: “[language[\_territory][.codeset][@modifier]]”. It affects the language in which Cinelerra will display menus and messages. To check your locale type locale in your terminal window. To see your available locales type locale -a.}

\nomenclature{\textbf{Lossless}}{term describing a compression method that allows the exact original data to be reconstructed from the compressed data, without any changes. Lossless compression is used for text and data files, but also for multimedia when quality is more important than file size.}

\nomenclature{\textbf{Lossy}}{term describing a compression method where compressing and then decompressing retrieves data that may well be different from the original, but is close enough to be useful in some way. Lossy compression is most commonly used to compress multimedia (audio, video, still images), especially for streaming. Repeatedly compressing and decompressing the file will cause, for most lossy compression formats, to progressively lose quality. MP3 is an example.}

\nomenclature{\textbf{Luma/Luminance}}{Luma represents the achromatic image without any color (like a black and white picture), the part of a video signal that includes information about its brightness. Luma is typically paired with chroma - luma represents the image without any color, while the chroma components represent the color information. Luma is designated with the letter Y.}

\nomenclature{\textbf{Media}}{generic term for audio, videos and images on some kind of storage.  This can include items as a short movie recorded on your phone, photos from your camera, MP3 songs, or a movie trailer.}

\nomenclature{\textbf{NLE}}{Non Linear Editing.  A modern editing method used by Cinelerra that records the decisions of the editor in an edit decision list (EDL) without modifying the original source files.}

\nomenclature{\textbf{NTSC}}{stands for National Television System Committee; a standard that defines a video with canvas size of 720x480 and a framerate of 29.97 fps.  This was originally based on United States broadcast television.}

\nomenclature{\textbf{On the fly}}{consumed as produced in real time.}

\nomenclature{\textbf{PAL}}{stands for Phase Alternating Line; a standard that defines a video with canvas size of 720x576 and a framerate of 25 fps. This is based on a European standard for broadcast television.}

\nomenclature{\textbf{Patch}}{an incremental change of any kind of program source.}

\nomenclature{\textbf{Patchbay}}{the area on the left of the timeline that contains the controls to enable features specific to each track.}

\nomenclature{\textbf{Picons}}{miniature images, also called thumbnails, of the video. In the Resources Window they represent the first frame of the asset. On the Timeline they imitate a physical film and are derived from the video data in the media file.}

\nomenclature{\textbf{Pillarbox}}{blacks bars on the top and bottom and/or left and right side of the frame. They are due to a smaller frame size than the one set in the project.}

\nomenclature{\textbf{Pixel}}{the smallest independent unit of a digital image; a minute area of illumination on a display screen, one of many from which an image is composed.  Word was invented from picture element.  In the 3D world, this is call a voxel.}

\nomenclature{\textbf{Plugin}}{program fragment that is loaded on demand as in “plugged in”.  In Cinelerra, these are often called “effects” and are used to provide a specific function on demand.}

\nomenclature{\textbf{Pop-up menu}}{a menu that pops up by clicking the right mouse button. Also called a context menu.}

\nomenclature{\textbf{Progressive scan}}{a method for capturing, storing, displaying or transmitting moving images in which the lines of each frame are drawn in sequence, in a path similar to text on a page - line by line, from top to bottom. It is in contrast to the “interlace” used in traditional television systems.}

\nomenclature{\textbf{Project}}{consists of the EDL, media, and other associated data objects to comprise the entire session for the purpose of rendering/creating the final result.}

\nomenclature{\textbf{Proxy file}}{a copy of an original media file but with low resolution or quality, used as temporary media for editing with lower CPU load or I/O. Rendering is then done with the high quality original.}

\nomenclature{\textbf{Raw image}}{an image file containing the unprocessed data from the image sensor of a digital camera or a scanner. Raw images from many different cameras can be loaded in Cinelerra. There are quite a wide-range variety of raw formats in existence.}

\nomenclature{\textbf{Record}}{the acquisition and storage of some media}

\nomenclature{\textbf{Render farm}}{a set of computers that work closely together for rendering.}

\nomenclature{\textbf{Rendering}}{the process of applying the instructions contained in the edit decision list (EDL) to produce an audio and/or video file. Also any processing of video, such as in rendering effects to the compositor.}

\nomenclature{\textbf{Resize}}{to reduce, enlarge or reshape the outline of an image, thus preserving the relative measures of the distances inside it.}

\nomenclature{\textbf{Resolution}}{the size of a digital image, width and height, measured in pixels.}

\nomenclature{\textbf{Resource(s)}}{the actual media and support items available for your projects which includes assets, clips, transitions and effects.  These are available in the Resources window when needed.}

\nomenclature{\textbf{RGB}}{(Red, Green, Blue) a color model in which red, green and blue lights are added together in various combinations to reproduce all the colors. RGBA is the same color model with extra information for transparency in the Alpha channel.  This mimics the receptors in the human eye – that is why we use it!}

\nomenclature{\textbf{Saturation}}{the intensity of a specific color. It measures the distance of a color from a neutral gray.}

\nomenclature{\textbf{Scale}}{to reduce or enlarge an image proportionally, preserving the ratio of the distances inside it. See resize.}

\nomenclature{\textbf{SEGV}}{Segmentation Fault Violation - a page fault that fails to read or write memory. For example, attempting to write to a read-only location, or to overwrite part of the operating system.}

\nomenclature{\textbf{Shell}}{a file containing a series of commands that provides an interface for users. In everyday use it indicates the Command Line.  Cinelerra has a “shell cmds” icon on the upper right corner of the main timeline where user written shell scripts can be added and easily accessed without exiting.}

\nomenclature{\textbf{Shot}}{in filmmaking and video production, a shot is a series of frames, that runs for an uninterrupted period of time.  In film editing, a shot is the continuous footage or sequence between two edits or cuts. Loosely used to refer to a single camera image, i.e. a shot.}

\nomenclature{\textbf{Solo}}{to activate only one function of a functional set.}

\nomenclature{\textbf{Source}}{a file containing media that has been saved on your operating system disk in a file.  Anything that provides data; origin of data.}

\nomenclature{\textbf{Splice}}{to unite edits by lapping the two ends together or by inserting an edit between two edits or in the middle of an edit.}

\nomenclature{\textbf{Split}}{to divide or break up an edit into two parts.}

\nomenclature{\textbf{Stream}}{a source of data that is usually accessed only sequentially, as in audio or video.}

\nomenclature{\textbf{Streaming}}{a method of receiving audio or video media while they are still being delivered, so that it is possible to watch video or listen to audio without waiting for an entire file to download. Streaming can be live or on-demand.}

\nomenclature{\textbf{Subtitle}}{the text of a video displayed on the bottom of the screen, often used for language translations. Subtitles are also the graphics displayed on top of video content, stored in a separate stream such as done for DVD menus.}

\nomenclature{\textbf{TOC}}{acronym for Table of Contents; this is one of several kinds of an index file used to accelerate media access.  It may contain compressed audio waveform data to accelerate timeline update.}

\nomenclature{\textbf{Telecine}}{a method of sampling media for both color and sample rate in order to prepare for presentation on a broadcast signal.  In the United States video is broadcast at 29.97 frames per second but film uses 24 frames per second. For the film's motion to be accurately rendered on the video signal, a telecine must use a technique called the 2:3 pulldown, to convert from film’s 24 frames per second to 29.97 frames per second.}

\nomenclature{\textbf{Thread}}{a single stream of program instruction fragments that may or may not be executed in parallel.}

\nomenclature{\textbf{Two screen editing}}{this refers to an editing method where you mark a portion of your source material with In and Out points and insert or overlay at a specific and marked point in your timeline, called the insertion point.}

\nomenclature{\textbf{Thumbnails}}{miniature images of the video. In the Resources Window they represent the first frame of the asset. When drawn on the timeline they imitate a physical film. See picons.}

\nomenclature{\textbf{Timebar}}{the part of the timeline that marks the time passing in selectable time units.}

\nomenclature{\textbf{Timeline}}{the part of the program window that contains video and audio tracks and displays the edits as they occur in time.}

\nomenclature{\textbf{Title}}{identifier applied to a data object; it is used to refer to the bar in the upper part of an asset that contains the name of the source file. It can be shown/hidden using the View menu. In the XML project file, TITLE is the name of the track, for example "Video 1".}

\nomenclature{\textbf{Title bar}}{presentation of a title for a given data object.}

\nomenclature{\textbf{Toggle}}{a program switch.}

\nomenclature{\textbf{Transitions}}{rendered output.}

\nomenclature{\textbf{Trimming}}{edit boundaries to lengthen or to shorten the duration of the edit in the timeline. Over the edit boundary and during the trimming operation the mouse pointer changes shape.}

\nomenclature{\textbf{Tumblers}}{a diamond shaped button composed of two arrows - one arrow upward and one arrow downward. It is used to set values in text boxes using the mouse either by clicking on the up/down arrow or using the wheel with the pointer over the tumbler.}

\nomenclature{\textbf{Tweaking}}{to make minor adjustments to.  Also, a general term from when automatic generation keyframes are armed.}

\nomenclature{\textbf{Underrun}}{a state occurring when a buffer used to communicate between two devices or processes is fed with data at a lower speed than the data is being read from it. This requires the program or device reading from the buffer to pause its processing while the buffer refills.}

\nomenclature{\textbf{Value}}{the measure of the brightness of a color. In video signals it is represented by luma.}

\nomenclature{\textbf{Vicons}}{stands for Video Icons, which are animated thumbnail presentations of video media.}

\nomenclature{\textbf{Waveform}}{the visual image of the form of the audio signal.}

\nomenclature{\textbf{Widget}}{a unitary graphical object that performs a specific set.  It is usually a single subwindow.}

\nomenclature{\textbf{XML}}{the language Cinelerra EDLs are written in. Extensible Markup Language (XML) is a general-purpose language that combines text and extra information about the text and allows users to make modifications. It creates a text representation of a data object that is designed to be relatively human-legible.}

\nomenclature{\textbf{YUV}}{is a color model which splits luma from chroma, similarly to human sight.. Colors are stored as absolute luma representation and the difference signal between luma and chroma complement components.  This color model is used by PAL and NTSC standards. Y stands for the luma component and U and V are the chroma components. U and V are actually color difference components (respectively R-Y and B-Y). In fact YUV signals are created from an original RGB source. The weighted values of R, G, and B are added together to produce a single Y signal, representing the overall luma. The U signal is then created by subtracting the Y from the blue signal of the original RGB, and then scaling; V is created by subtracting the Y from the red, and then scaling by a different factor. Previous black-and-white systems used only luma (Y) information and color information (U and V) was added so that a black-and-white receiver would still be able to display a color picture as a normal black and white picture.}

\nomenclature{\textbf{8 / 10-bit images}}{images that were sampled at 8 or 10 bits per channel. The tone range for each channel is from 0 to 255 for 8-bit and from 0 to 1023 for 10-bit.}

\nomenclature{\textbf{Algorithm}}{set of instructions, typically to solve a class of problems or perform a computation.}

\nomenclature{\textbf{Aliasing}}{incorrect sampling of a video or audio signal that leads to unwanted artifacts because it cannot distinguish values too close. There is Temporal Aliasing (audio) and Spatial Aliasing (video).}

\nomenclature{\textbf{B-spline}}{(basis spline) polynomial curves characterized by nodes and control points to obtain smooth curves.}

\nomenclature{\textbf{Banding}}{Incorrect display of a color gradient, due to a narrow tonal range, which leads to the appearance of color bands instead of fading tones.}

\nomenclature{\textbf{Bayer array}}{is a color filter array upon a sensor to get an RGB image. The filter pattern is 50\% green, 25\% red and 25\% blue.}
    
\nomenclature{\textbf{Bicubic filter}}{is a mathematical interpolation to resample an image. It produces a smoother result than the nearest-neighbor or the bilinear filter.}

\nomenclature{\textbf{Bilinear filter}}{is a mathematical interpolation to resample an image.  It produces a smoother result than the nearest-neighbor.}

\nomenclature{\textbf{Bit depth}}{Quantization of an audio or video signal. Color depth is the number of bits used for each color component of a single pixel. Audio bit depth is the number of bits of information in each sample.}
    
\nomenclature{\textbf{Black point}}{the part of the image with the darkest value that can be displayed on a device. At the limit is 0 (pure black).}

\nomenclature{\textbf{Color space}}{organization of the colors of a color model, limited to the gamut of a particular device. Examples are sRGB and rec 709.}

\nomenclature{\textbf{Color timing}}{the process of adjusting color balance making it consistent in every scene. We also talk about color matching scene to scene or shot matching.}

\nomenclature{\textbf{Composite}}{the combination of two or more video layers to obtain a single composition.}

\nomenclature{\textbf{Contrast}}{is the difference in luminance of parts or elements of an image that makes them distinguishable. The greater the difference, the greater the contrast. It is the horizontal range shown by the histogram or the vertical range shown by the waveform.}

\nomenclature{\textbf{Denoise}}{is the process of removing digital noise from a video. The three main types are based on statistical methods, transform wavelets and temporal averaging.}

\nomenclature{\textbf{DeSpill}}{process to remove background color contamination from the edges of the subject in foreground, during a chroma key.}

\nomenclature{\textbf{Digital Intermediate (DI)}}{Over time it has taken on different meanings. For Cinelerra GG is meant as the creation of a high quality file that during the various stages of editing and color correction keeps as much information as possible. Being little or uncompressed its manipulation is also faster and more efficient.}

\nomenclature{\textbf{Dynamic range}}{is the ratio between the largest and smallest values (luminance) that an image can assume. The larger the size, the better we can distinguish details in the dark and light parts.}

\nomenclature{\textbf{Exposure}}{the exposure is the amount of light per unit area reaching an image sensor, as determined by shutter speed, lens aperture and scene luminance.}

\nomenclature{\textbf{Floating point}}{real numbers with decimals. They allow for greater precision in calculations than integers, but they require more processing power.}

\nomenclature{\textbf{Gamut}}{In color reproduction, the gamut is a certain complete subset of colors. The larger the gamut of a device (associated with a color space) the more colors can be displayed.}

\nomenclature{\textbf{HDR images}}{are images that have a dynamic range larger than that of the sensor used. They are created directly with a rendering or with the merge of several images at low dynamic range. They require the use of floating points.}

\nomenclature{\textbf{HDTV}}{(high definition TV) standard characterized by a 16:9 aspect ratio, various frames rates and scan modes and with a resolution of at least 1080.}

\nomenclature{\textbf{Lanczos}}{algorithm for high quality resampling video signal. It is also used in case of upsampling, weak point of other similar filters.}

\nomenclature{\textbf{Letterbox}}{blacks bars on the top and bottom side of the frame. They are due to a smaller frame size than the one set in the project (see also Pillarbox).}

\nomenclature{\textbf{LUT, 3D LUT}}{(LookUp Table) used to map one color space to another. Cinelerra GG uses them through ffmpeg filters. There are downloadable collections or there are specific ones provided by hardware manufacturers.}

\nomenclature{\textbf{Nearest neighbor}}{is the simplest method of resampling an image. It is fast and resources saving, but produces less smooth results.}

\nomenclature{\textbf{Panning}}{in video technology, panning refers to the horizontal scrolling of an image wider than the display. In Cinelerra it is done (together with other camera movements) with the camera tool.}

\nomenclature{\textbf{Pillarbox}}{blacks bars on the left and right side of the frame. They are due to a smaller frame size than the one set in the project (see also Letterbox).}

\nomenclature{\textbf{Rec 709}}{(BT.709 or ITU 709) is the standard color space of high-definition television, having 16:9 aspect ratio, scan modes and frame rate of HDTV. It can decode at 8 or 10 bits per channel. The gamut is the same as for sRGB from which it differs for the 2.4 gamma.}

\nomenclature{\textbf{Retiming}}{is the change of the speed of a edit by interpolating the original frames to a new in between frame.}

\nomenclature{\textbf{sRGB}}{is an RGB color space used on monitors, printers, and the Internet. It can decode at 8 or 10 bits per channel. The gamut is the same as for Rec 709 from which it differs for the 2.2 gamma.}

\nomenclature{\textbf{Subsampling}}{is the practice of encoding Y’CbCr images by implementing less resolution for chroma information than for luma information. Depending on the amount of information deleted we have: 4:4:4 (lossless); 4:2:2 (HDTV); 4:2:0 (DVD, smartphone) or 4:1:1.}

\nomenclature{\textbf{Timecode}}{is a sequence of numeric codes generated at regular intervals by a timing synchronization system and recorded into audio and/or video tracks. It is used for synchronization audio and video clips.}

\nomenclature{\textbf{White point}}{is the clearest part of an image that can be displayed on a device; at the limit is 1.0 value (pure white).}

\nomenclature{\textbf{YCbCr}}{is a color space based on the YUV color model, widely used in broadcast productions. It separates the luma part (Y) from the chroma part (Cb and Cr).}





